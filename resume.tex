\documentclass{resume}

\newcommand{\en}[1]{#1}
\newcommand{\zh}[1]{}

\zh{\usepackage{xeCJK}}
\zh{\setCJKmainfont[Path=/Users/shaonhuang/Library/Fonts/]{SourceHanSerif.ttc}}
\zh{\setCJKsansfont[Path=/Users/shaonhuang/Library/Fonts/]{SourceHanSansSC-VF.otf}}
\zh{\setCJKmonofont[Path=/Users/shaonhuang/Library/Fonts/]{SourceHanSansSC-VF.otf}}

\begin{document}

\name{\en{Yukuai Huang}\zh{黄宇快}}
\basicInfo{
      \faMobilePhone{ +86 188-1001-5082} \textperiodcentered\
      \email{yukuaihuang@icloud.com} \textperiodcentered\
      \github[shaonhuang]{https://github.com/shaonhuang}
}

\section{\en{Education}\zh{教育经历}}

\en{\datedsubsection{\textbf{Beijing University of Posts and Telecommunications}, Bachelor's Degree}{09/2018 -- 06/2022}}
\zh{\datedsubsection{\textbf{北京邮电大学}, 本科}{2018/09 -- 2022/06}}
\begin{itemize}
      \item \en{Major: Digital media technology}
            \zh{数字媒体技术}
      \item \en{Key Courses: Data structure, Database, Network, User interface design, Web Front-End, Web Back-End, Mobile application development}
            \zh{主修课程:数据结构、数据库技术与应用、计算机网路、用户界面设计、动态图形设计、计算机图形学、Web前端技术、Web后端技术、移动应用开发}
\end{itemize}

\section{\en{Work Experience}\zh{工作经历}}

\en{\datedsubsection{\textbf{\href{https://intl.cloud.baidu.com}{Baidu AI Cloud}}, Beijing, China}{01/2021 -- 08/2021}}
\zh{\datedsubsection{\textbf{\href{https://cloud.baidu.com/}{百度智能云(Baidu AI Cloud)}}}{2021/01 -- 2021/08}}
\en{\role{ACG Web construction and Cloud marketing department}{Front-End Intern}}
\zh{\role{ACG建站与云市场部}{前端研发实习}}
\begin{itemize}
      \item \en{Responsible for Baidu Intelligent Cloud-Cloud Monitoring BCM, which is a monitoring product based on Baidu Cloud and provides 7*24 hours of alarm monitoring service. My work is to migrate the old version front-end interface to the new version of the development work, maintain the iteration and upgrade of functions, responsible for site monitoring of the new version, application monitoring, etc., and rebuild to San framework based on Er framework page.}
            \zh{负责百度智能云-云监控BCM,BCM是依托于百度云的监控产品,提供7*24小时的报警监控服务,本人的工作是对老版本前端界面迁移至新版本的开发工作,维护功能的迭代和升级,负责过新版站点监控,应用监控等,基于Er框架页面重构至San框架。}
      \item \en{On the development side, we design and optimize an optional table which is strongly related to the requirement when we meet the requirement that component libraries cannot be used. Over the past six months, through internships, with relevant personnel such as PM and UE, RD, QA, etc., to achieve inter-departmental DevOps and agile development, to standardize code and enhance comprehensive engineering capabilities, and to improve understanding between development and enterprise business.}
            \zh{在开发方面,遇到无法用组件库的需求,自己设计并优化出一个与需求强相关的可选表格。在过去半年的时间内,通过实习,配合PM和UE,RD,QA等相关人员实现部门间的DevOps和敏捷开发,规范代码并提升综合工程能力,在开发与企业业务之间的理解上也得到提升。}
      \item \en{Responsible for Baidu Intelligent Cloud-Video Creation Distribution Platform VOD, migrate old BUI component library to SUI component library. During the development, by understanding the business, find experience problems in the process, and problems caused by different style effects of the old component library on different browsers, timely feedback and repair.}
            \zh{负责百度智能云-视频创作分发平台VOD,迁移老式BUI组件库至SUI组件库,在开发时,通过了解业务,发现流 程上的体验问题,以及老式组件库对不同浏览器的样式效果不同导致的问题,及时反馈和修复。}
\end{itemize}

\section{\en{Portfolios}\zh{个人项目}}
\datedsubsection{\textbf{ProteinRecorder}}{\url{https://www.digitalcreak.top/video?frg=1}}
\en{Android Application}
\zh{安卓应用}
\begin{itemize}
      \item \en{Proteinrecorder is a software for recording daily protein intake. It uses kotlin language to realize the overall process framework and code writing of the app.}
            \zh{ProteinRecorder 是记录日常蛋白质摄入量的软件,用Kotlin语言实现app的整体流程框架和代码编写。}
      \item \en{Use lxml package to crawl data, store data in JSON format, read and build SQLite database, fuzzy search database to complete the search function, efficient data addition and deletion list, and display it in recyclerlistview through database call, which effectively improves the operation efficiency of the system.}
            \zh{使用 lxml 包爬取数据,json格式存储数据,读取并构建SQLite数据库,模糊搜索数据库以完成搜索功能, 高效的数据添加和删除列表,通过数据库的调用显示到RecyclerListView中,有效的提升了系统的运行效率。}
\end{itemize}

\datedsubsection{\textbf{Bouyei Website}}{\url{http://www.digitalcreak.top:18080/}}
\en{Minority Introduction FE Preject}
\zh{少数民族 Web 前端项目}
\begin{itemize}
      \item \en{Use Vue cli framework to realize, be familiar with CSS + div page layout, and use vuejs to realize data-driven view.}
            \zh{使用vue-cli框架来实现,熟悉CSS+DIV页面布局,用Vuejs实现数据驱动视图。}
      \item \en{Write templates of different modules to realize progressive development, and generate a large number of articles through fixed templates.}
            \zh{编写不同模块的template来实现渐进式开发,通过固定模板实现大量文章的生成。}
      \item \en{Design and write basic reusable components, such as rotation map, search bar, etc.}
            \zh{设计并编写基础可复用的组件,如轮播图,搜索栏等。}
\end{itemize}

\datedsubsection{\textbf{Geeks}}{\url{https://www.digitalcreak.top/video?frg=3}}
\en{Mobile Application UI/UE Preject}
\zh{极客网”交互设计方案}
\begin{itemize}
      \item \en{Complete the design of "geek network" alone, implement it with sketch according to the design sample, and finally meet the design requirements to complete a relatively complete set of interactive logic and experience.}
            \zh{独自完成”极客网”的设计,并按照设计样稿,采用Sketch进行实现,最终符合设计要求完成一套比较 完整的交互逻辑和体验。}
\end{itemize}

\datedsubsection{\textbf{快宇Blog}}{\url{https://www.digitalcreak.top/}}
\en{Personal Website}
\zh{个人网站}

\section{\en{Skills}\zh{技能}}
\begin{itemize}[parsep=0.25ex]
      \item \en{\textbf{Design Tools}:
                  Photoshop,Adobe Illustrator,Maya, Sketch}
            \zh{\textbf{设计软件}:
                  Photoshop,Adobe Illustrator,Maya,Sketch}
                  
      \item \en{\textbf{Programming Languages}:
                  not limited to any specific language,
                  and experienced in Python,
                  comfortable with JavaScript/TypeScript/Java (in random order).}
            \zh{\textbf{编程语言}:
                  不局限于特定编程语言,且尤其熟悉 JavaScript 等,
                  了解 Java/Python/Kotlin/TypeScript 等(不分先后)。}

      \item \en{\textbf{Developing Tools}:
                  experienced in Linux-based programming,
                  have experience with team tools like Git, etc.}
            \zh{\textbf{开发工具}:
                  十分熟悉 Linux,有 Git 等团队合作工具的经验。}
\end{itemize}

\end{document}
