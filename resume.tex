\documentclass{resume}

\newcommand{\en}[1]{#1}
\newcommand{\zh}[1]{}

\zh{\usepackage{xeCJK}}
\zh{\setCJKmainfont[Path=/Users/shaonhuang/Library/Fonts/]{SourceHanSerif.ttc}}
\zh{\setCJKsansfont[Path=/Users/shaonhuang/Library/Fonts/]{SourceHanSansSC-VF.otf}}
\zh{\setCJKmonofont[Path=/Users/shaonhuang/Library/Fonts/]{SourceHanSansSC-VF.otf}}

\begin{document}

\name{\en{Shaon Huang}\zh{黄宇快}}
\basicInfo{
      \faMobilePhone{ +86 188-1001-5082} \textperiodcentered\
      \email{yukuaihuang@icloud.com} \textperiodcentered\
      \github[shaonhuang]{https://github.com/shaonhuang}
}

\section{\en{Education}\zh{教育经历}}

\en{\datedsubsection{\textbf{Beijing University of Posts and Telecommunications}, Bachelor's Degree}{09/2018 -- 06/2022}}
\zh{\datedsubsection{\textbf{北京邮电大学}, 本科}{2018/09 -- 2022/06}}
\begin{itemize}
      \item \en{Major: Digital media technology}
            \zh{数字媒体技术}
      \item \en{Key Courses: Data structure, Database, Network, User interface design, Web Front-End, Web Back-End, Mobile application development}
            \zh{主修课程:数据结构、数据库技术与应用、计算机网路、用户界面设计、动态图形设计、计算机图形学、Web前端技术、Web后端技术、移动应用开发}
\end{itemize}

\section{\en{Work Experience}\zh{工作经历}}

\en{\datedsubsection{\textbf{\href{https://intl.cloud.baidu.com}{Baidu AI Cloud}}, Beijing, China}{01/2021 -- 08/2021}}
\zh{\datedsubsection{\textbf{\href{https://cloud.baidu.com/}{百度智能云(Baidu AI Cloud)}}}{2021/01 -- 2021/08}}
\en{\role{ACG Web construction and Cloud marketing department}{Front-End Intern}}
\zh{\role{ACG建站与云市场部}{前端研发实习}}
\begin{itemize}
      \item \en{Responsible for Baidu Intelligent Cloud-Cloud Monitoring BCM webpage, which is a monitoring product based on Baidu Cloud service and provides 7*24 hours of warning monitoring service. My duty is to migrate the early release(MVC) front-end interface to the new version is based on the MVVM model. Keep maintaining the product's iteration and upgrading functions. In charge of the new version website monitoring webpage, application monitoring webpage, etc. Perfected user experience, and enhanced single page capability.}
            \zh{负责百度智能云-云监控BCM,BCM是依托于百度云的监控产品,提供7*24小时的报警监控服务,本人的工作是对老版本前端界面迁移至新版本的开发工作,维护功能的迭代和升级,负责过新版站点监控,应用监控等,基于Er框架页面重构至San框架。提升用户体验,同时提高单页面应用交互能力。}
      \item \en{On the development side, I design and optimize an HTML table component that is strongly related to the requirement, which can't be acquired with the template table component. Over the past six months internships, with relevant personnel such as PM and UE, RD, QA, etc. I achieve inter-departmental DevOps and agile development. standardize code and enhance comprehensive programming capabilities, and improve understanding between development and enterprise business.}
            \zh{在开发方面,遇到无法用组件库的需求,自己设计并优化出一个与需求强相关的可选表格。在过去半年的时间内,通过实习,配合PM和UE,RD,QA等相关人员实现部门间的DevOps和敏捷开发,规范代码并提升综合工程能力,在开发与企业业务之间的理解上也得到提升。}
      \item \en{Responsible for Baidu Intelligent Cloud-Video Creation Distribution Platform VOD. Migrating old BUI component library to SUI component library. During the development, by understanding the business, find interaction experience issues in the process, and problems caused by different CSS style effects on the old component library. Feedbacking to PM, and repair.}
            \zh{负责百度智能云-视频创作分发平台VOD,迁移老式BUI组件库至SUI组件库,在开发时,通过了解业务,发现流 程上的体验问题,以及老式组件库对不同浏览器的样式效果不同导致的问题,及时反馈和修复。}
\end{itemize}

\en{\datedsubsection{\textbf{{Institute of Automation,Chinese Academy of Sciences}}, Beijing, China}{2022/03 -- 2022/04}}
\zh{\datedsubsection{\textbf{{中科院自动化所}}}{2022/03 -- 2022/04}}
\en{FE Confidential project}
\zh{前端保密项目(外包)}
\begin{itemize}
      \item \en{Web pages are based on Vue3, Vue-CLI, and Vuex with LocalStorage to maintain data persistence, use Sass to make theme be changed, Express framework to build up a mockup platform, FE and BE development keeps paralleling maks project process steps by steps.}
            \zh{页面基于vue3,vue-cli,vuex搭配localstorage实现数据持久化,sass实现风格转换,express搭建mockup前端平台,前后端分离开发,快速迭代。}
      \item \en{In charge of the front-end development of the project, feeds the need for feature to make a custom float block which can hover and click to navigate to detail pages. Import Echart components to visualize data, and improve user experience.}
            \zh{负责前端页面开发,自定义悬浮方块组件并多处复用,满足内容导向功能,引入Echart图表,提高部分页面可视化功能,提高用户体验,解决用户痛点。}
      \item \en{Overcome the short development time, and complicated component functions (some simple components can use Element components, but 90\% of the components needs to design by myself), efficient communication online, solving problems each by each during development, tested, and delivered.}
            \zh{克服开发时间较短(2周),页面功能复杂(部分采用Element组件,页面功能90\%不可使用现有组件库),有效线上沟通,解决页面研发全流程问题,并测试且交付。}
\end{itemize}

\section{\en{Portfolios}\zh{个人项目}}
\datedsubsection{\textbf{V2rayX}}{\url{https://github.com/shaonhuang/V2rayX}}
\en{Electron Application}
\zh{Electron Application}
\begin{itemize}
      \item \en{An all-platform (Macos Windows LInux) V2ray client build with electron.}
            \zh{跨平台的Electron 应用,用于管理并创建V2ray服务,实现网络代理。}
      \item \en{Using technologies stacks such as Tailwind CSS, Material-UI, React, Redux, Electron, Electron-Vite, and Node.js, I developed a powerful client for managing services. By managing configuration files, configuring operating system proxies, and establishing PAC service and image scanning to import services, I created a powrful user interface with excellent interactivity.}
            \zh{使用 Tailwindcss、Material-UI、React、Redux、Electron、Electron-Vite 和 Node.js 等技术,开发了一个功能强大的服务管理工具客户端。我通过管理配置文件、配置操作系统代理、建立 PAC 和图片扫描导入等服务,实现了交互良好的用户界面。}
\end{itemize}

\datedsubsection{\textbf{ProteinRecorder}}{\url{https://resume-shaonhuang.oss-us-west-1.aliyuncs.com/proteinrecorder2.mp4}}
\en{Android Application}
\zh{安卓应用}
\begin{itemize}
      \item \en{Proteinrecorder is a software for recording daily protein intake. It uses kotlin language to achieve the overall process framework and code writing of the app.}
            \zh{ProteinRecorder 是记录日常蛋白质摄入量的软件,用Kotlin语言实现app的整体流程框架和代码编写。}
      \item \en{Use lxml package to crawl data, store data in JSON format, read and build SQLite database, fuzzy search database to complete the search function, efficient data addition and deletion list, and display it in recyclerlistview through database call, which effectively improves the operation efficiency of the system.}
            \zh{使用 lxml 包爬取数据,json格式存储数据,读取并构建SQLite数据库,模糊搜索数据库以完成搜索功能, 高效的数据添加和删除列表,通过数据库的调用显示到RecyclerListView中,有效的提升了系统的运行效率。}
\end{itemize}

\datedsubsection{\textbf{Bouyei Website}}{}
\en{Minority Introduction FE Preject}
\zh{少数民族 Web 前端项目}
\begin{itemize}
      \item \en{Using vuejs to realize data-driven view. Writing templates of different modules to realize progressive development, and generate a large number of articles through fixed templates.}
            \zh{Vuejs实现数据驱动视图,编写不同模块的template来实现渐进式开发,通过固定模板实现大量文章的生成。}
      \item \en{Design and write basic reusable components, such as Draw Marquee, search bar, etc.}
            \zh{设计并编写基础可复用的组件,如轮播图,搜索栏等。}
\end{itemize}

\datedsubsection{\textbf{Geeks}}{\url{https://resume-shaonhuang.oss-us-west-1.aliyuncs.com/geek_demo.mp4}}
\en{Mobile Application UI/UE Preject}
\zh{极客网”交互设计方案}
\begin{itemize}
      \item \en{Complete the design of "geek network" alone, implement it with sketch according to the design sample, and finally meet the design requirements to complete a relatively complete set of interactive logic and experience.}
            \zh{独自完成”极客网”的设计,并按照设计样稿,采用Sketch进行实现,最终符合设计要求完成一套比较 完整的交互逻辑和体验。}
\end{itemize}

\section{\en{Skills}\zh{技能}}
\begin{itemize}[parsep=0.25ex]
      \item \en{\textbf{Programming Languages}:
                  not limited to any specific language,
                  and experienced in Python,
                  comfortable with JavaScript/TypeScript/Java (in random order).}
            \zh{\textbf{编程语言}:
                  不局限于特定编程语言,且尤其熟悉 JavaScript 等,
                  了解 Java/Python/Kotlin/TypeScript 等(不分先后)。}

      \item \en{\textbf{Developing Tools}:
                  experienced in Linux-based programming,
                  have experience with team tools like Git, etc.}
            \zh{\textbf{开发工具}:
                  十分熟悉 Linux,有 Git 等团队合作工具的经验。}
\end{itemize}

\end{document}
