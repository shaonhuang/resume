\documentclass{resume}

\newcommand{\en}[1]{#1}
\newcommand{\zh}[1]{}

\zh{\usepackage{xeCJK}}
\zh{\setCJKmainfont[Path=/Users/shaonhuang/Library/Fonts/]{SourceHanSerif.ttc}}
\zh{\setCJKsansfont[Path=/Users/shaonhuang/Library/Fonts/]{SourceHanSansSC-VF.otf}}
\zh{\setCJKmonofont[Path=/Users/shaonhuang/Library/Fonts/]{SourceHanSansSC-VF.otf}}

\begin{document}

\name{\en{Shaon Huang}\zh{黄宇快}}
\basicInfo{
      \faMobilePhone{ +86 188-1001-5082} \textperiodcentered\
      \email{yukuaihuang@icloud.com} \textperiodcentered\
      \github[shaonhuang]{https://github.com/shaonhuang}
}

\section{\en{Education}\zh{教育经历}}

\en{\datedsubsection{\textbf{\href{\detokenize{https://en.wikipedia.org/wiki/Beijing_University_of_Posts_and_Telecommunications}}{Beijing University of Posts and Telecommunications}}, Bachelor's Degree}{09/2018 -- 06/2023}}
\zh{\datedsubsection{\textbf{\href{\detokenize{https://baike.baidu.com/item/北京邮电大学/139535}}{北京邮电大学}}, 本科}{2018/09 -- 2023/06}}
\begin{itemize}
      \item \en{Major: Computer Science and Technology – Digital Media Technology}
            \zh{专业:计算机科学与技术-数字媒体技术}
      \item \en{Key Courses: Data Structures, Database Technology and Applications, Computer Networks, User Interface Design, Motion Graphics Design, Computer Graphics, Web Front-End Development, Web Back-End Development, Mobile Application Development}
            \zh{主修课程:数据结构、数据库技术与应用、计算机网络、用户界面设计、动态图形设计、计算机图形学、Web前端技术、Web后端技术、移动应用开发}
\end{itemize}

\section{\en{Work Experience}\zh{工作经历}}

% 百度智能云
\en{\datedsubsection{\textbf{\href{https://intl.cloud.baidu.com}{Baidu AI Cloud}}, Beijing, China}{01/2021 -- 08/2021}}
\zh{\datedsubsection{\textbf{\href{https://cloud.baidu.com/}{百度智能云(Baidu AI Cloud)}}}{2021/01 -- 2021/08}}
\en{\role{ACG Web Construction and Cloud Marketing Department}{Front-End Intern}}
\zh{\role{ACG 建站与云市场部}{前端研发实习}}
\begin{itemize}
    \item \en{Led development of Baidu AI Cloud-BCM providing 24×7 cloud product monitoring and alerting services. Responsible for migrating legacy frontend interfaces to new versions, maintaining feature iterations and upgrades, and developing new site monitoring and application monitoring modules. Refactored pages from Er framework (MVC) to San framework (MVVM), improving user experience and enhancing single-page application interactivity.}
          \zh{负责百度智能云-云监控 BCM,提供 7x24 小时的云产品报警监控服务,负责对老版本前端界面迁移至新版本的开发,维护功能的迭代和升级,负责新版站点监控,应用监控等,基于Er 框架(MVC)页面重构至 San (MVVM)框架。提升用户体验,同时提高单页面应用交互能力。}
    \item \en{Owned Baidu AI Cloud-VOD (Video Creation \& Distribution) migration from legacy BUI to SUI, proactively addressing UX workflow issues and cross-browser style inconsistencies to deliver a markedly more consistent user experience.}
          \zh{负责百度智能云-视频创作分发平台 VOD,迁移老式 BUI 组件库至 SUI 组件库,主动提出流程上的体验问题,以及老式组件库对不同浏览器的样式效果不同导致的问题,及时反馈和修复,用户体验一致性显著提升。}
\end{itemize}

% Absotlute IT
\en{\datedsubsection{\textbf{Absotlute IT}, Australia}{2024/01 -- Present}}
\zh{\datedsubsection{\textbf{Absotlute IT}, 澳大利亚}{2024/01 -- 至今}}
\en{\role{INDOS - Independent Online Solutions}{Full Stack Engineer}}
\zh{\role{INDOS - Independent Online Solutions(线上超市快送系统)}{全栈工程师}}
\begin{itemize}
    \item \en{Implemented legacy system features using EJS/Express stack, including unit testing frameworks, coupon management system, scheduled POS file upload automation, and Google Analytics \& Matomo integration, increasing monthly active user monitoring coverage by 80\%. Integrated Sentry for crash monitoring, reducing average error response time by 96\%.}
          \zh{负责旧版的功能开发 - 例如单元测试,优惠券系统,POS文件的上传的定时任务系统,集成 Google Analytics \& Matomo,月活用户监测覆盖率提升 80\%。集成 Sentry 实现崩溃监控,平均错误响应时间缩短 96\%。}
    \item \en{Led migration from legacy stack to modern architecture with Remix+React frontend and Encore(Rust) backend. Leveraged AI capabilities to accomplish 86\% of frontend features and 40\% of backend services migration within one month, equivalent to 2.5 FTE workload. Key challenges: maintaining consistent high-quality code output and refactoring scattered EJS business logic and service code distributed across multiple files.}
          \zh{主导 Remix+React 前端与 encore(Rust) 后端迁移 - 借助AI能力快速实现86\%的前端功能以及40\%的后端服务的迁移(一个月)。工作量相当于2.5人。难点:持续输出高质量代码以及重构分散在各处的ejs业务逻辑代码和服务代码。}
\end{itemize}

% Chad
\en{\datedsubsection{\textbf{\href{https://www.trychad.com/}{Chad}}}{}}
\zh{\datedsubsection{\textbf{\href{https://www.trychad.com/}{Chad}}}{}}
\en{\role{Front-End Engineer}{}}
\zh{\role{前端工程师}{}}
\begin{itemize}
    \item \en{Tech stack: Remix (React), Shadcn UI, Radix UI, Firebase, Framer Motion, Tailwind CSS, Zustand}
          \zh{技术栈:Remix(React)、Shadcn UI、Radix UI、Firebase、Framer Motion、Tailwind CSS、Zustand}
    \item \en{Developed frontend modules including navigation bars and account settings pages, collaborating with backend team to implement Firebase JWT authentication and resolve related authentication requirements.}
          \zh{负责前端模块开发,包括导航栏、账户设置页等,配合后端通过JWT解决 Firebase 认证问题等需求。}
    \item \en{Implemented pixel-perfect animations and custom components from Figma designs, unifying interaction language to improve user retention and build brand trust.}
          \zh{实现Figma中 1:1 动画效果和自定义组件,统一交互语言,提升用户留存率,构建品牌信任。}
    \item \en{Built modular frontend pages using the above tech stack, designing and implementing backend data-driven pages to increase system flexibility, such as a backend-driven navigation bar.}
          \zh{使用上述技术栈构建前端页面,模块化设计,设计并实现后端数据驱动的页面,提高系统灵活度。如后端数据驱动导航栏。}
\end{itemize}

\section{\en{Portfolios}\zh{个人项目}}

% V2rayX Electron
\en{\datedsubsection{\textbf{\href{https://github.com/shaonhuang/V2rayX/tree/v0.4.5}{V2rayX (Electron Version)}}}{}}
\zh{\datedsubsection{\textbf{\href{https://github.com/shaonhuang/V2rayX/tree/v0.4.5}{跨平台代理客户端 – V2rayX(Electron 版本)}}}{}}
\begin{itemize}
    \item \en{Designed and developed a cross-platform proxy client built on Electron, supporting Windows, macOS, and Linux. Integrated multiple proxy protocols (VMess, VLESS, Trojan) with support for various proxy modes. Achieved 1,100+ daily active users on Snap platform and 16,794+ total GitHub downloads.}
          \zh{设计并实现基于 Electron 的跨平台代理客户端,支持 Windows、macOS、Linux,集成多种代理协议(VMess、VLESS、Trojan),支持全局、PAC、规则等多种代理模式,提供节点订阅与二维码识别导入功能。Snap平台日活用户1100+,GitHub总下载数16,794+(\href{https://tooomm.github.io/github-release-stats/?username=shaonhuang&repository=V2rayX}{统计链接})}
    \item \en{Built a modular and scalable frontend architecture with React, MUI, Vite, and Tailwind CSS. Integrated Sentry for crash monitoring, reducing error response time by 40\%.}
          \zh{使用 React、MUI、Vite 和 Tailwind 构建前端架构,模块化设计,便于维护和扩展。集成 Sentry 实现 Crash 监控,错误响应时间缩短 40\%。}
    \item \en{Developed a fully automated CI/CD pipeline with GitHub Actions, supporting multi-platform packaging (deb, dmg, exe, etc.), significantly improving development efficiency.}
          \zh{构建 GitHub Actions CI/CD 流水线,自动构建多平台安装包(deb、dmg、exe、etc),提高开发效率与交付速度。}
\end{itemize}

% V2rayX Rust/Tauri
\en{\datedsubsection{\textbf{\href{https://github.com/shaonhuang/V2rayX}{V2rayX (Rust/Tauri 2 Refactor)}}}{}}
\zh{\datedsubsection{\textbf{\href{https://github.com/shaonhuang/V2rayX}{跨平台代理客户端 – V2rayX(Rust/Tauri 2 重构版本)}}}{}}
\begin{itemize}
    \item \en{Led the migration of V2rayX from Electron to a modern Rust + Tauri architecture, achieving substantial performance improvements. Reduced code duplication by 40\%, improved startup speed by 2.3x, decreased memory usage by 75.2\%, and minimized installer size by 76.74\%.}
          \zh{\href{\detokenize{https://shaonhuang.vercel.app/posts/v2rayx-refactor}}{将 Electron 版本迁移至 Tauri + Rust 架构},显著提升应用性能和稳定性,减少资源占用。代码重复率减少40\%。应用启动速度提升 2.3倍,内存占用减少 75.2\%(161.4M -> 40M)。安装包体积下降约76.74\%。}
    \item \en{Applied advanced Rust paradigms for type safety and error handling, eliminating 80\% of redundant exception logic and increasing code coverage to 85\%.}
          \zh{利用 Rust 的类型安全和错误处理机制设计思想,减少重复的异常处理代码,提高代码质量和可维护性。删除 80\% 重复异常检测代码,代码覆盖率提高至 85\%。}
    \item \en{Implemented comprehensive UI internationalization (10+ languages) and major UX enhancements including dark/light themes and activity logging. Built with Remix architecture using TypeScript, ESLint, and Husky for code quality.}
          \zh{增加 UI 国际化与 UX 设计,支持十种以上语言(中、英、波斯等)和暗黑/浅色主题;实现活动日志、服务分享等实用增强功能,基于 Remix 全栈架构,采用 TypeScript、ESLint、Husky 保证代码质量,后端逻辑使用 Rust 实现安全性能兼顾。}
\end{itemize}

\section{\en{Skills}\zh{技能}}
\begin{itemize}[parsep=0.25ex]
      \item \en{\textbf{Full-Stack Engineering}: Comprehensive understanding of the complete product engineering lifecycle with hands-on experience in frontend technologies (React, Remix) and backend development (Node.js, Rust). Proficient in build tooling, CI/CD automation, and cloud deployment strategies. Language-agnostic approach with deep expertise in JavaScript, TypeScript, and Rust—highly adaptable to diverse tech stacks and project requirements.}
            \zh{\textbf{全栈工程能力}: 对产品全流程工程有深入理解,具备前端(如 React、Remix)和后端(如 Node.js、Rust)开发经验,熟悉构建环境、CI/CD 自动化与云端部署。不局限于特定语言,尤为熟悉 JavaScript、TypeScript、Rust,具备多语言开发能力,能够灵活适应不同技术栈和项目需求。}
      \item \en{\textbf{Development Tools \& Practices}: Expert-level proficiency in Linux development environments, agile development methodologies, DevOps practices, and comprehensive testing strategies. Extensive collaborative development experience with advanced Git workflow management for efficient project coordination.}
            \zh{\textbf{开发工具与实践}: 精通 Linux 环境开发,熟练掌握敏捷开发流程、DevOps 实践及单元测试。团队协作经验丰富,擅长使用 Git 等工具高效管理项目。}
\end{itemize}

\end{document}
